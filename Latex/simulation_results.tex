\documentclass[9pt, xcolor=table]{beamer}

\usepackage[utf8]{inputenc}
\usepackage[round, comma]{natbib}
\usepackage{amsmath}
\usepackage{hyperref}
\usepackage{amsfonts}
\usepackage{verbatim}
\DeclareMathOperator*{\argmin}{arg\,min}



\mode<presentation> {

\usetheme{Madrid}

\setbeamertemplate{navigation symbols}{} 
\useinnertheme{circles}
\definecolor{greenish}{RGB}{0, 153, 76}
\usecolortheme[named=greenish]{structure}
}

\setbeamertemplate{headline}
{%
  \begin{beamercolorbox}[ht=3.5ex,dp=1.125ex,%
      leftskip=.3cm,rightskip=.3cm plus1fil]{section in head/foot}
    \usebeamerfont{section in head/foot}\usebeamercolor[fg]{section in head/foot}%
\insertsectionnavigationhorizontal{\paperwidth}{\hskip0pt plus1fill}{\hskip0pt plus1fill}
  \end{beamercolorbox}%
  \begin{beamercolorbox}[colsep=1.5pt]{middle separation line head}
  \end{beamercolorbox}
  \begin{beamercolorbox}[colsep=1.5pt]{lower separation line head}
  \end{beamercolorbox}
}

\title[Interpretation of black box models]{Interpretation of black box models using tree-based surrogate models \newline \small{Simulations}}
\author[Sofia Loibl]{Sofia Loibl}
\institute[LMU]{LMU München}
\date{\today}

\begin{document}

\begin{frame}
\titlepage 
\end{frame}


\begin{frame}
\frametitle{Outline} 
\tableofcontents 
\end{frame}


\section{Simulation basic Scenarios}
\begin{frame}{Simulation design}
Comparison of four MBT algorithms (SLIM, GUIDE, MOB, CTree)
with respect to performance, stability and interpretability.

\vspace{0.3cm}
\begin{itemize}
    \item 3 basic scenarios (linear smooth, linear abrupt, linear mixed)
    \item MBT as standalone (to measure accuracy of the MBT algorithms), surrogate for lm and surrogate for xgboost model (to measure fidelity)
    \item 3 different sample sizes (1000, 5000, 10000)
    \item 3 different pruning parameters (for alpha or impr)\end{itemize}
    
$\Rightarrow 3 \cdot 3 \cdot 3 \cdot 3 = 81$ Experiments for each MBT algorithm

\vspace{0.3cm}
100 Simulation runs

    
\end{frame}

\begin{frame}{Evaluation measures}
\begin{enumerate}
    \item Performance: $R^2$ and $MSE$ on training and test data
    \item Interpretability: number of leaf nodes
    \item Stability: Adjusted Rand Index (ARI)
\end{enumerate}
    
\end{frame}

\section{Linear Smooth}
\begin{frame}{Linear Smooth}
\begin{itemize}
    \item $x_1,..., x_3 \sim U(-1,1)$ $\epsilon \sim N(0, sd_{data})$
    \item $ f_{ls}(x) = x_1 + 4   x_2 + 3   x_2   x_3 $
    \item $\epsilon \sim N(0, 0.1 sd(f_{ls}(x))$
    \item $y = f(x) + \epsilon$
\end{itemize}

\end{frame}

\begin{frame}{Linear Smooth}

\begin{table}
\caption{Mean simulation results on 100 simulation runs for \textbf{SLIM} as stand alone model on scenario Linear smooth with n = 1000 for different values of impr}
\centering \small
\begin{tabular}[t]{r|r|r|r|r}
\hline
impr & number of leave nodes & R2 train & R2 test & ARI\\
\hline
0.15 & 2.27 & 0.9586746 & 0.9560847 & 0.4734244\\
\hline
0.10 & 10.08 & 0.9856378 & 0.9830595 & 0.2669574\\
\hline
0.05 & 14.57 & 0.9908326 & 0.9884542 & 0.3258697\\
\hline
\end{tabular}
\end{table}


\begin{table}

\caption{Mean simulation results on 100 simulation runs for \textbf{GUIDE} as stand alone model on scenario Linear smooth with n = 1000 for different values of impr}
\centering \small
\begin{tabular}[t]{r|r|r|r|r}
\hline
impr & number of leave nodes & R2 train & R2 test & ARI\\
\hline
0.15 & 2.25 & 0.9584099 & 0.9559713 & 0.5213677\\
\hline
0.10 & 9.70 & 0.9855840 & 0.9830494 & 0.2939543\\
\hline
0.05 & 14.49 & 0.9906753 & 0.9883918 & 0.3220653\\
\hline
\end{tabular}
\end{table}
    
\end{frame}




\begin{frame}{Linear Smooth}
\begin{table}

\caption{Mean simulation results on 100 simulation runs for \textbf{MOB} as stand alone model on scenario Linear smooth with n = 1000 for different values of alpha}
\centering \small
\begin{tabular}[t]{r|r|r|r|r}
\hline
alpha & number of leave nodes & R2 train & R2 test & ARI\\
\hline
0.001 & 9.65 & 0.9898598 & 0.9876318 & 0.3472862\\
\hline
0.010 & 10.96 & 0.9902735 & 0.9879067 & 0.3403858\\
\hline
0.050 & 12.51 & 0.9905938 & 0.9882748 & 0.3691474\\
\hline
\end{tabular}
\end{table} 


\begin{table}

\caption{Mean simulation results on 100 simulation runs for \textbf{CTree} as stand alone model on scenario Linear smooth with n = 1000 for different values of alpha}
\centering \small
\begin{tabular}[t]{r|r|r|r|r}
\hline
impr & number of leave nodes & R2 train & R2 test & ARI\\
\hline
0.001 & 11.51 & 0.9900921 & 0.9881395 & 0.3763795\\
\hline
0.010 & 12.89 & 0.9904182 & 0.9884063 & 0.3657001\\
\hline
0.050 & 13.75 & 0.9905168 & 0.9885340 & 0.3632570\\
\hline
\end{tabular}
\end{table}
\end{frame}

\begin{frame}{Linear Smooth}
Results:
\begin{itemize}
    \item All four algorithms achieve good performance
    \item The number of leafnodes is high considering that the data generating process is very simple and involves only one interaction 
    \item with a similar number of leaf nodes (i.e. approximately the same level of interpretability), MOB and GUIDE provide more stable MBTs and have a better performance
    \item SLIM and CTree are very sensitive regarding the choice of impr
\end{itemize}
    
\end{frame}

\begin{frame}{Linear Smooth}

    
\end{frame}


\begin{frame}{Linear abrupt}
\begin{itemize}
    \item $x_1, x_2 \sim U(-1,1)$, $x_3 \sim Bern(0.5)$
    \item $ f_{la}(x) = x_{1} - 8  x_2 + 16  x_2  \mathbf{I}_{x_3 = 0} + 8  x_2  \mathbf{I}_{x_1 > mean(x_1)}$
    \item $\epsilon \sim N(0, 0.1 sd(f_{la}(x))$
    \item $y = f(x) + \epsilon$
\end{itemize}    
\end{frame}


\begin{frame}{Bibliography}
    \bibliography{bibliography}
    \bibliographystyle{dcu}

\end{frame}
\end{document}
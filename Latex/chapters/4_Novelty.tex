Currently three different possible approaches:
\subsection{Unbiased SLIM Alternatives}
Avoid selection bias by using MOB, CTree or GUIDE to split.
Comparison of the methods in \citep{Schlosser.24.06.2019}

\subsubsection{MOB}
Score based M-Fluctuation Test \\
\textbf{Advantage}: High Power in detecting structural change points
\subsubsection{CTree}
Score based Permutation Test \\
\textbf{Advantage}: High Power in detecting smooth changes

\subsubsection{GUIDE}
Residual based categorical association test\\
\textbf{Advantage}: Scores do not have to be available\\
\textbf{Disadvantage}: Less Power than the other tests




\subsection{SLIM with Permutation Tests}
\textbf{Idea}: \citep{.4581}
Test at each split by permutation tests whether there is a significant relationship between the predictor and the response.\\
\textbf{Approach}: Hold the predictors constant and permute the reponse variable. Generate Models on permuted datasets and determine frequency of models equal or better than that with the original data\\

\subsection{Stable SLIM}
Improve the stability by applying a method similar to \citep{Zhou.2018}\\
\textbf{Approach}: Test for each alternative split candidate if the difference in Gini gains is greater than zero. The sum of p-values of this test gives a bound of likelihood of  splitting the current node a different way. Subsample data to control this probability
\textbf{Disadvantage}:  Computationally very complex


\subsection{SLIM combined with concepts}
\citep{Renard.2019}

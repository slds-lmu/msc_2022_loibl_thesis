Surrogate models are one possibility to explain complex black box models post-hoc. According to \citet{Molnar.2019} the purpose of (interpretable) surrogate models is to approximate the predictions of the underlying model as accurately as possible and to be interpretable at the same time.
Depending on whether the goal is to achieve a global interpretation of a black box model (model explanation) or only to explain the results of individual input instances (outcome explanation), global or local surrogate models can be used \citep{Maratea.2021}.
An advantage of using surrogate models as interpretable machine learning (IML) -method is, that it is a model-agnostic approach, i.e., it does not require any information about the inner workings of the black box model \citep{Molnar.2019}.

%\subsection{Global surrogate models}

The concept of global surrogate models is very simple. To explain the underlying model, the predictions for some input data are first calculated using this black box model. Then, an arbitrarily interpretable surrogate model is trained on the input data and the predictions. Finally, it can be measured how well the surrogate model reproduces the black-box predictions \citep{Molnar.2019}. If the performance is good enough, the interpretation can be made according to the chosen surrogate model.

\citet{Molnar.2019} cites the flexibility in choosing the surrogate model and the simplicity of the approach as advantages of this method. However, this flexibility is also a challenge, as the choice of an appropriate surrogate model is a trade-off between high performance due to the potentially higher complexity of the model class and its interpretability. For example, linear models are relatively easy to interpret, but cannot uncover nonlinear relationships that may have been modeled in the underlying black box model. \citet{Molnar.2019} provides an overview of interpretable models and their strengths and weaknesses, such as generalized additive models (GAM), decision trees, or sparse linear models (e.g. lasso regression). 

In addition, other promising high-performance models that can be interpreted to a certain degree are being developed. Generalized additive models plus interactions (GA2M) of \citet{Lou.2013} are an example of an extension of generalized additive models that allow a small number of two-way interactions to be efficiently modeled in addition to the nonlinear main effects.


One aspect that must be remembered when using global surrogate models is that only conclusions about the underlying black box model can be drawn, not about the actual data generating process \citep{Molnar.2019}.

%\subsection{Local surrogate model}

If only a single prediction is to be explained instead of the entire model, local interpretable model-agnostic explanations (LIME) can be used \citep{Ribeiro.2016}. Instead of having to find a (possibly highly complex) interpretable model on the entire feature space, only a surrogate model for a small neighborhood around the instance of interest is trained. This is realized by observing how the predictions of the black-box model behave when small variations of the instance of interest are introduced into the model. Like global surrogates, LIME is very flexible in the choice of an appropriate surrogate model. The advantage over global surrogate models is that it is easier to find a good compromise between model complexity and interpretability, since the model only needs to fit a very small range of the feature space. Even sparse main effect models could imitate the black box model well in the small area of the feature space, for example. One difficulty, however, is choosing an appropriate neighborhood for the instance of interest to which the model is to be fitted.
\citep{Molnar.2019}


%\subsection{Sub-regional surrogate models}
A approach that tries to combine the advantages of global and local surrogate methods, i.e., to obtain models that are easy to understand on the one hand, but at the same time cover the entire feature space without sacrificing performance, is summarized here under the term subregional surrogate model.
The idea is to subdivide the feature space into appropriate subregions where fitting easy-to-interpret models is sufficient.
With K-LIME (K local interpretable model-agnostic explanations), \citet{Hall.2017}  pursue the approach of unsupervised partitioning of the feature space into K subregions using a K-means clustering algorithm. 
SLIM \citep{Hu.2020}, on the other hand, uses a supervised approach (MBT) to subdivides the feature space according to a given objective. According to \citep{Hu.2018}, SLIM is superior to K-LIME, which is mainly because K-LIME does not include the objective in the clustering. 
Another recent approach, called maidrr (Model-Agnostic Interpretable Data-driven suRRogate) is presented in \citet{Henckaerts.2022}. There the feature space is partitioned on the basis of partial dependence plots and then linear models in categorical format are fitted in the subregions. The categorial format is intended to increase interpretability, but thereby has a negative effect on the performance.

In this thesis, the SLIM algorithm is used and compared with other algorithms that can create MBTs.


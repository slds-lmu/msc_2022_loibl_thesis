\subsection{Evaluation measures}
on training and test data as well as on the original targets (accuracy) and predictions of different black box models (fidelity)
\subsubsection{Performance}
\begin{itemize}
    \item MSE
    \item R squared
\end{itemize}

\subsubsection{Stability}
\begin{itemize}
    \item Assumption: All methods have a problem with instability \citep{Fokkema.2020}
    \item Definition of stability? A distinction can be made between semantic and structural stability \citep{Wang.2018}
    \item how similar are the found subspaces? (decision region compatibility) \citep{Wang.2018}

    \item difference in tree size and depth can be used as indicator \citep{Wang.2018}


\end{itemize}






\subsubsection{Interpretability}
According to \citet{DoshiVelez.2017} Interpretability is deifined as ability to explain or to present in understandable
terms to a human

Number of leafnodes to reach the same performance

\subsection{Linear Models in the leafnodes}
Comparison of SLIM, MOB, CTree and GUIDE

Simulation Scenarios:
\begin{itemize}
    \item numerical or/and categorical features
    \item abrupt vs smooth vs mixed interactions
    \item different correlation patterns between features
\end{itemize}


\subsection{Sparse (spline-based) models in the leafnodes}
Comparison of SLIM and GA2M (EBM)
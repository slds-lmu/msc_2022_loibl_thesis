\subsection{Evaluation measures}
on training and test data as well as on the original targets (accuracy) and predictions of different black box models (fidelity)
\subsubsection{Performance}
\begin{itemize}
    \item MSE
    \item R squared
\end{itemize}

\subsubsection{Stability}

\begin{itemize}
    \item Assumption: All methods have a problem with instability \citep{Fokkema.2020}
    \item Definition of stability? A distinction can be made between semantic and structural stability \citep{Wang.2018}
    \item how similar are the found subspaces? (decision region compatibility) \citep{Wang.2018}

    \item difference in tree size and depth can be used as indicator \citep{Wang.2018}


\end{itemize}






\subsubsection{Interpretability}
According to \citet{DoshiVelez.2017} Interpretability is deifined as ability to explain or to present in understandable
terms to a human

Number of leafnodes to reach the same performance


\subsection{Simulation Scenarios}

\subsubsection{Basic scenarios}
\textbf{Linear smooth}
Numerical features with linear effects on y and smooth interactions


\textbf{Linear abrupt}
Numerical and binary features with linear effects and abrupt interactions

\textbf{Linear mixed}
Numerical and binary features with linear effects, abrupt and smooth interactions

\subsubsection{Correlated features}

\subsubsection{Big number of noisy features}
Linear mixed + many noise variables
SLIM and GUIDE with LASSO
MOB and CTree standard lm

\subsubsection{Non-linear Effects}
\begin{itemize}
    \item Linear mixed with non-linear main effects
    \item Linear mixed with non-linear main effect + nonlinear interactions
\end{itemize}

SLIM and GUIDE with GAMs and penalized polynomial regression


